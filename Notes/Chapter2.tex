\documentclass[en, oneside]{vivi}

\ProjectInfos{Introduction to Quantum Mechanics}{PHY-512}{Spring, 2025}{Chapter 2}{Time-Independent Schrödinger Equation}{Vivi}[https://github.com/Vivi26499]{24S153073}

\begin{document}

\section{Stationary States}
In the Schrödinger equation,
\begin{equation}
    i\hbar\frac{\partial \Psi}{\partial t} = -\frac{\hbar^2}{2m}\frac{\partial^2 \Psi}{\partial x^2} + V\Psi,
\end{equation}
we assume that the potential energy $V$ is time-independent, i.e., $V = V(x)$, in which case the wave function $\Psi(x, t)$ can be separated into two parts:
\begin{equation}
    \Psi(x, t) = \psi(x) \varphi(t).
\end{equation}
For separated wave functions, we have
\begin{align}
    \frac{\partial \Psi}{\partial t} &= \psi \frac{\dif \varphi}{\dif t}, \\
    \frac{\partial^2 \Psi}{\partial x^2} &= \frac{\dif[2] \psi}{\dif x^2} \varphi,
\end{align}
and the Schrödinger equation becomes
\begin{equation}
    i\hbar \psi \frac{\dif \varphi}{\dif t} = -\frac{\hbar^2}{2m} \frac{\dif[2] \psi}{\dif x^2} \varphi + V\psi\varphi.
\end{equation}
Dividing both sides by $\psi\varphi$, we get
\begin{equation}
    i\hbar \frac{1}{\varphi} \frac{\dif \varphi}{\dif t} = -\frac{\hbar^2}{2m} \frac{1}{\psi} \frac{\dif[2] \psi}{\dif x^2} + V.
\end{equation}
Since the left-hand side of the equation depends only on $t$ and the right-hand side depends only on $x$, they must be equal to a constant, which we denote as $E$:
\begin{equation}
    i\hbar \frac{1}{\varphi} \frac{\dif \varphi}{\dif t} = E,
\end{equation}
or
\begin{equation} \label{eq:2.8}
    \frac{\dif \varphi}{\dif t} = -\frac{iE}{\hbar} \varphi,
\end{equation}
and
\begin{equation}
    -\frac{\hbar^2}{2m} \frac{1}{\psi} \frac{\dif[2] \psi}{\dif x^2} + V = E,
\end{equation}
or
\begin{equation} \label{eq:2.10}
    \boxed{
        -\frac{\hbar^2}{2m} \frac{\dif[2] \psi}{\dif x^2} + V\psi = E\psi.
    }
\end{equation}
Then we have turned the Schrödinger equation, a partial differential equation, into two ordinary differential equations, equation \eqref{eq:2.8} and equation \eqref{eq:2.10}.\\
The solution to equation \eqref{eq:2.8} is
\begin{equation}
    \varphi(t) = e^{-iEt/\hbar},
\end{equation}
as we care only about the product $\psi(x)\varphi(t)$.\\
The equation \eqref{eq:2.10} is the time-independent Schrödinger equation.\\
There are three important properties of the solutions to equation \eqref{eq:2.10}:
\begin{enumerate}
    \item They are stationary states. Although the wave function itself,
    \begin{equation} \label{eq:stationary state}
        \Psi(x, t) = \psi(x) e^{-iEt/\hbar},
    \end{equation}
    oscillates in time, the probability density,
    \begin{align}
        |\Psi(x, t)|^2 &= \Psi^* \Psi\\
        &= \psi^* e^{iEt/\hbar} \psi e^{-iEt/\hbar}\\
        &= |\psi(x)|^2,
    \end{align}
    is time-independent. The same thing happens when calculating the expectation value of any dynamical variable,
    \begin{align}
        \langle Q(x, p) \rangle &= \int \Psi^* \left[Q(x, -i \hbar \frac{\dif}{\dif x})\right] \Psi dx\\
        &= \int \psi^* \left[Q(x, -i \hbar \frac{\dif}{\dif x})\right] \psi dx
    \end{align}
    is constant in time. In particular, $\langle x \rangle$ is constant in time, hence $\langle p \rangle = \frac{\dif \langle x \rangle}{\dif t} = 0$.
    \item They are states of definite energy. In classical mechanics, the total energy (kinetic plus potential) is called the Hamiltonian:
    \begin{equation}
        H(x, p) = \frac{p^2}{2m} + V(x),
    \end{equation}
    which is corresponded to the Hamiltonian operator, by replacing $p$ with $-i\hbar \frac{\partial}{\partial x}$:
    \begin{equation}
        \hat{H} = -\frac{\hbar^2}{2m} \frac{\partial^2}{\partial x^2} + V(x).
    \end{equation}
    Using Hamiltonian operator, the time-independent Schrödinger equation (equation \eqref{eq:2.10}) can be written as
    \begin{equation}
        \hat{H} \psi = E \psi.
    \end{equation}
    The expectation value of the total energy is
    \begin{align}
        \langle H \rangle &= \int \Psi^* \hat{H} \Psi \dif x\\
        &= \int \psi^* \hat{H} \psi \dif x\\
        &= \int \psi^* E \psi \dif x\\
        &= E \int |\psi|^2 \dif x\\
        &= E,
    \end{align}
    and
    \begin{align}
        \langle H^2 \rangle &= \int \Psi^* \hat{H}^2 \Psi \dif x\\
        &= \int \psi^* \hat{H}^2 \psi \dif x\\
        &= \int \psi^* E^2 \psi \dif x\\
        &= E^2 \int |\psi|^2 \dif x\\
        &= E^2.
    \end{align}
    So the variance of $H$ is
    \begin{equation}
        \sigma_H^2 = \langle H^2 \rangle - \langle H \rangle^2 = E^2 - E^2 = 0,
    \end{equation}
    which means that every measurement of the total energy is certain to return the value $E$.
    \item The general solution is a linear combination of separated solutions (stationary states).
    The time-independent Schrödinger equation (equation \eqref{eq:2.10}) yields an infinite collection of solutions, $\{ \psi_n(x) \}$, 
    each with its associated separation constant, $\{ E_n \}$; thus there is a different wave function for each allowed energy:
    \begin{equation}
        \Psi_n(x, t) = \psi_n(x) e^{-iE_nt/\hbar}.
    \end{equation}
    The general solution is a linear combination of these solutions:
    \begin{align}
        \Psi(x, t) &= \sum_{n=1}^\infty c_n \Psi_n(x, t)\label{eq:general solution}\\ 
        &= \sum_{n=1}^\infty c_n \psi_n(x) e^{-iE_nt/\hbar},
    \end{align}
    where the coefficients $\{ c_n \}$ can be chosen to satisfy the initial conditions:
    \begin{equation}
        \Psi(x, 0) = \sum_{n=1}^\infty c_n \psi_n(x).
    \end{equation}
    $| c_n |^2$ is the probability that a measurement of the energy would return to the value $E_n$. Thus of course,
    \begin{equation}
        \sum_{n=1}^\infty | c_n |^2 = 1,
    \end{equation}
    and the expectation value of the energy is
    \begin{equation}
        \langle E \rangle = \sum_{n=1}^\infty | c_n |^2 E_n.
    \end{equation}
\end{enumerate}

\section{The Infinite Square Well}
The infinite square well potential is defined as
\begin{equation}
    V(x) = \begin{cases}
        0, & 0 \leq x \leq a,\\
        \infty, & \text{otherwise}.
    \end{cases}
\end{equation}
Outside the well, $\psi(x) = 0$. Inside the well, the time-independent Schrödinger equation (equation \eqref{eq:2.10}) becomes
\begin{equation}
    -\frac{\hbar^2}{2m} \frac{\dif[2] \psi}{\dif x^2} = E\psi,
\end{equation}
or
\begin{equation}
    \frac{\dif[2] \psi}{\dif x^2} = -k^2 \psi,
\end{equation}
where $k = \frac{\sqrt{2mE}}{\hbar}$. The general solution is
\begin{equation}
    \psi(x) = A \sin kx + B \cos kx.
\end{equation}
Continuity of $\psi(x)$ requires that
\begin{equation}
    \psi(0) = 0,
\end{equation}
which means $B = 0$, and
\begin{equation}
    \psi(a) = 0,
\end{equation}
which means
\begin{equation}
    k a = 0, \pm \pi, \pm 2\pi, \cdots
\end{equation}
But $k = 0$ is trivial and the negative solutions give nothing new, so the distinct solutions are
\begin{equation}
    k_n = \frac{n\pi}{a}, \quad n = 1, 2, 3, \cdots
\end{equation}
Hence the possible values of $E$ are
\begin{equation}
    \boxed{
        E_n = \frac{\hbar^2 k_n^2}{2m} = \frac{n^2 \pi^2 \hbar^2}{2m a^2}.
    }
\end{equation}
To find $A$, we normalize $\psi(x)$:
\begin{align}
    1 &= \int_0^a |\psi(x)|^2 \dif x\\
    &= \abs{A}^2\int_0^a \sin^2 kx \dif x\\
    &= \abs{A}^2 \frac{a}{2},
\end{align}
which means $A = \sqrt{\frac{2}{a}}$. Then the solutions are
\begin{equation}
    \psi_n(x) = \sqrt{\frac{2}{a}} \sin \left(\frac{n\pi}{a} x\right).
\end{equation}
As a collection, the functions $\{ \psi_n(x) \}$ have some interesting and important properties:
\begin{enumerate}
    \item They are alternately even and odd: $\psi_1(x)$ is odd, $\psi_2(x)$ is even, $\psi_3(x)$ is odd, and so on.
    \item As $n$ increases, each successive state has one more node: $\psi_1(x)$ has none, $\psi_2(x)$ has one, $\psi_3(x)$ has two, and so on.
    \item They are mutually orthonormal, in the sense that
    \begin{equation}
        \int \psi_m(x)^* \psi_n(x) \dif x = \delta_{nm},
    \end{equation}
    where $\delta_{nm}$ is the Kronecker delta.
    \item They are complete, in the sense that any function $f(x)$ can be expressed as a linear combination of them:
    \begin{align}
        f(x) &= \sum_{n=1}^\infty c_n \psi_n(x)\\
        &= \sqrt{\frac{2}{a}} \sum_{n=1}^\infty c_n \sin \left(\frac{n\pi}{a} x\right),
    \end{align}
    where $c_n$ can be evaluated by
    \begin{align} 
        \int \psi_m(x)^* f(x) \dif x &= \int \psi_m(x)^* \left(\sum_{n=1}^\infty c_n \psi_n(x)\right) \dif x\label{eq:complete1}\\
        &= \sum_{n=1}^\infty c_n \int \psi_m(x)^* \psi_n(x) \dif x\\
        &= \sum c_n \delta_{mn}\\
        &= c_m\label{eq:complete4}.
    \end{align}
\end{enumerate}
Actually, the first is true whenever $V(x)$ is symmetric; the second is true for any $V(x)$; the third and fourth are true for any $V(x)$.\\
The stationary states (equation \eqref{eq:stationary state}) of the infinite square well are
\begin{align}
    \Psi_n(x, t) &= \psi_n(x) e^{-iE_nt/\hbar}\\
    &= \sqrt{\frac{2}{a}} \sin \left(\frac{n\pi}{a} x\right) e^{-i(n^2 \pi^2 \hbar/2ma^2)t}.
\end{align}
Then the general solution (equation \eqref{eq:general solution}) is
\begin{align}
    \Psi(x, t) &= \sum_{n=1}^\infty c_n \Psi_n(x, t)\\
    &= \sum_{n=1}^\infty c_n \sqrt{\frac{2}{a}} \sin \left(\frac{n\pi}{a} x\right) e^{-i(n^2 \pi^2 \hbar/2ma^2)t}.
\end{align}
The coefficients $\{ c_n \}$ can be determined by the initial condition $\Psi(x, 0)$, using equation \eqref{eq:complete1} to \eqref{eq:complete4}.

\section{The Harmonic Oscillator}
The potential energy of a classical harmonic oscillator is
\begin{equation}
    V(x) = \frac{1}{2} k x^2,
\end{equation}
with the solution $x(t) = A \sin(\omega t) + B \cos(\omega t)$, where $\omega = \sqrt{k/m}$.\\
The quantum problem is to solve the Schrödinger equation for the potential
\begin{equation}
    V(x) = \frac{1}{2} m \omega^2 x^2.
\end{equation}
The time-independent Schrödinger equation (equation \eqref{eq:2.10}) becomes
\begin{equation} \label{eq:harmonic oscillator}
    -\frac{\hbar^2}{2m} \frac{\dif[2] \psi}{\dif x^2} + \frac{1}{2} m \omega^2 x^2 \psi = E\psi.
\end{equation}

\subsection{Algebraic Method}
Using the momentum operator $\hat{p} = -i\hbar \frac{\dif}{\dif x}$, equation \eqref{eq:harmonic oscillator} can be written as
\begin{equation}
    \frac{1}{2m} \left[ \hat p^2 + (m \omega x)^2 \right] \psi = E\psi.
\end{equation}
The basic idea is to factor the Hamiltonian,
\begin{equation}
    \hat H = \frac{1}{2m} \left[ \hat p^2 + (m \omega x)^2 \right],
\end{equation}
with two ladder operators,
\begin{equation} \label{eq:ladder operators}
    \hat a_\pm = \frac{1}{\sqrt{2\hbar m \omega}} \left( \mp i \hat p + m \omega x \right).
\end{equation}
The commutator of two operators $\hat A$ and $\hat B$ is
\begin{equation}
    [\hat A, \hat B] = \hat A \hat B - \hat B \hat A.
\end{equation}
The commutator of $\hat x = x$ and $\hat p = -i\hbar \frac{\dif}{\dif x}$ can be calculated as
\begin{align}
    [\hat x, \hat p] f(x) &= \hat x \hat p f(x) - \hat p \hat x f(x)\\
    &= x \left( -i\hbar \frac{\dif}{\dif x} f(x) \right) - \left( -i\hbar \frac{\dif}{\dif x} \right) (x f(x))\\
    &= -i\hbar \left[ x \frac{\dif f(x)}{\dif x} - x \frac{\dif f(x)}{\dif x} - f(x) \right]\\
    &= i\hbar f(x),
\end{align}
which means $[\hat x, \hat p] = i\hbar$.\\
The product $\hat a_- \hat a_+$ is
\begin{align}
    \hat a_- \hat a_+ &= \frac{1}{2\hbar m \omega} \left( i\hat p + m \omega x \right) \left( -i\hat p + m \omega x \right)\\
    &= \frac{1}{2\hbar m \omega} \left( \hat p^2 + (m \omega x)^2 - i m \omega (x \hat p - \hat p x) \right)\\
    &= \frac{1}{2\hbar m \omega} \left( \hat p^2 + (m \omega x)^2 - i m \omega [\hat x, \hat p] \right)\\
    &= \frac{1}{2\hbar m \omega} \left( \hat p^2 + (m \omega x)^2 + \hbar m \omega \right)\\
    &= \frac{1}{\hbar \omega} \hat H + \frac{1}{2},
\end{align}
thus $\hat H = \hbar \omega \left( \hat a_- \hat a_+ - \frac{1}{2} \right)$.\\
Similarly, the product $\hat a_+ \hat a_-$ is
\begin{equation}
    \hat a_+ \hat a_- = \frac{1}{\hbar \omega} \hat H - \frac{1}{2},
\end{equation}
thus $\hat H = \hbar \omega \left( \hat a_+ \hat a_- + \frac{1}{2} \right)$.\\
In terms of $\hat a_\pm$, the Schrödinger equation (equation \eqref{eq:harmonic oscillator}) becomes
\begin{equation} \label{eq:harmonic oscillator 2}
    \hbar \omega \left( \hat a_\pm \hat a_\mp \pm \frac{1}{2} \right) \psi = E\psi.
\end{equation}
\begin{thm}
    If $\psi$ satisfies the Schrödinger equation with energy $E$, i.e.,
    \begin{equation}
        \hat H \psi = E\psi,
    \end{equation}
    then $\hat a_\pm \psi$ satisfies the Schrödinger equation with energy $E \pm \hbar \omega$, i.e.,
    \begin{equation}
        \hat H \hat a_\pm \psi = (E \pm \hbar \omega) \hat a_\pm \psi.
    \end{equation}
\end{thm}

\begin{pf}
    \begin{align}
        \hat H \hat a_\pm \psi &= \hbar \omega \left( \hat a_\pm \hat a_\mp \pm \frac{1}{2} \right) \hat a_\pm \psi\\
        &= \hbar \omega \left( \hat a_\pm \hat a_\mp \hat a_\pm \pm \frac{1}{2} \hat a_\pm \right) \psi\\
        &= \hbar \omega \hat a_\pm \left( \hat a_\mp \hat a_\pm \pm \frac{1}{2} \right) \psi\\
        &= \hat a_\pm \hbar \omega \left( \hat a_\pm \hat a_\mp \pm 1 \pm \frac{1}{2} \right) \psi\\
        &= \hat a_\pm (\hat H \pm \hbar \omega) \psi\\
        &= (E \pm \hbar \omega) \hat a_\pm \psi.
    \end{align}
\end{pf}
That's why $\hat a_+$ is called the raising operator and $\hat a_-$ is called the lowering operator.\\
The ground state of the harmonic oscillator is the state annihilated by $\hat a_-$:
\begin{equation}
    \hat a_- \psi_0 = 0.
\end{equation}
Substituting $\hat a_-$ with its expression, we get
\begin{align}
    \frac{1}{\sqrt{2\hbar m \omega}} \left( i \hat p + m \omega x \right) \psi_0 &= 0\\
    \left( \hbar \frac{\dif}{\dif x} + m \omega x \right) \psi_0 &= 0\\
    \frac{\dif \psi_0}{\dif x} &= -\frac{m \omega}{\hbar} x \psi_0,
\end{align}
which is a first-order differential equation. The solution is
\begin{equation}
    \psi_0(x) = A e^{- \frac{m \omega}{2\hbar} x^2}.
\end{equation}
The normalization condition is
\begin{align}
    1 &= \int_{-\infty}^{+\infty} |\psi_0(x)|^2 \dif x\\
    &= |A|^2 \int_{-\infty}^{+\infty} e^{- \frac{m \omega}{\hbar} x^2} \dif x\\
    &= |A|^2 \sqrt{\frac{\pi \hbar}{m \omega}},
\end{align}
which means $A = \left( \frac{m \omega}{\pi \hbar} \right)^{1/4}$, and hence
\begin{equation}
    \psi_0(x) = \left( \frac{m \omega}{\pi \hbar} \right)^{1/4} e^{- \frac{m \omega}{2\hbar} x^2}.
\end{equation}
To determine the energy of the ground state, we plug it into the Schrödinger equation (equation \eqref{eq:harmonic oscillator 2}) and note that $\hat a_+ \psi_0 = 0$:
\begin{equation}
    \hbar \omega \left( \hat a_- \hat a_+ - \frac{1}{2} \right) \psi_0 = E_0 \psi_0,
\end{equation}
thus $E_0 = \frac{1}{2} \hbar \omega$.\\
Then the excited states can be obtained by applying the raising operator to the ground state, increasing the energy by $\hbar \omega$ each time:
\begin{equation} \label{eq:harmonic oscillator excited states}
    \psi_n = A_n (\hat a_+)^n \psi_0, \quad E_n = \left( n + \frac{1}{2} \right) \hbar \omega,
\end{equation}
where $A_n$ is the normalization constant.\\
We know that $\hat a_\pm \psi_n$ is proportional to $\psi_{n \pm 1}$,
\begin{equation}
    \hat a_+ \psi_n = c_n \psi_{n + 1}, \quad \hat a_- \psi_n = d_n \psi_{n - 1}.
\end{equation}
\begin{thm}
    For "any" functions $f(x)$ and $g(x)$, 
    \begin{equation}
        \int_{-\infty} ^\infty f ^* (\hat a_\pm g) \dif x = \int_{-\infty} ^\infty (\hat a_\mp f) ^* g \dif x.
    \end{equation}
\end{thm}
\begin{pf}
    \begin{align*}
        \int_{-\infty} ^\infty f ^* (\hat a_\pm g) \dif x &= \frac{1}{\sqrt{2\hbar m \omega}} \int_{-\infty} ^\infty f ^* (\mp \hbar \frac{\dif}{\dif x} + m \omega x) g \dif x\\
        &= \frac{1}{\sqrt{2\hbar m \omega}} \left[ \mp \hbar f ^* g \bigg|_{-\infty} ^\infty + \int_{-\infty} ^\infty \left[\left(\pm \hbar \frac{\dif}{\dif x} + m \omega x\right) f\right] ^* g \dif x \right]\\
        &= \int_{-\infty} ^\infty (\hat a_\mp f) ^* g \dif x.
    \end{align*}
\end{pf}
In particular,
\begin{equation}
    \int_{-\infty} ^\infty (\hat a_\pm \psi_n) ^* (\hat a_\pm \psi_n) \dif x = \int_{-\infty} ^\infty (\hat a_\mp \hat a_\pm \psi_n) ^* \psi_n \dif x.
\end{equation}
From equation \eqref{eq:harmonic oscillator 2} and equation \eqref{eq:harmonic oscillator excited states}, we have
\begin{align}
    \hbar \omega \left( \hat a_\pm \hat a_\mp \pm \frac{1}{2} \right) \psi_n &= E_n \psi_n\\
    &= \hbar \omega \left( n + \frac{1}{2} \right) \psi_n,
\end{align}
which means
\begin{equation}
    \hat a_+ \hat a_- \psi_n = n \psi_n, \quad \hat a_- \hat a_+ \psi_n= (n + 1)\psi_n,
\end{equation}
so,
\begin{align}
    \int_{-\infty} ^\infty (\hat a_+ \psi_n) ^* (\hat a_+ \psi_n) \dif x &= |c_n|^2 \int_{-\infty} ^\infty \psi_{n + 1} ^* \psi_{n + 1} \dif x\\
    &= (n+1) \int_{-\infty} ^\infty \psi_n ^* \psi_n \dif x,
\end{align}
so $|c_n|^2 = n + 1$.\\
Similarly, we can get $|d_n|^2 = n$.\\
Hence,
\begin{equation}
    \hat a_+ \psi_n = \sqrt{n + 1} \psi_{n + 1}, \quad \hat a_- \psi_n = \sqrt{n} \psi_{n - 1}.
\end{equation}
Using mathematical induction, we have
\begin{equation}
    \psi_n = \frac{1}{\sqrt{n!}} (\hat a_+)^n \psi_0.
\end{equation}
As in the case of the infinite square well, the stationary states of the harmonic oscillator are orthonormal:
\begin{equation}
    \int_{-\infty} ^\infty \psi_m ^* \psi_n \dif x = \delta_{mn}.
\end{equation}
\begin{pf}
    \begin{align*}
        n \int_{-\infty} ^\infty \psi_m ^* \psi_n \dif x &= \int_{-\infty} ^\infty \psi_m ^* \hat a_+ \hat a_- \psi_n \dif x\\
        &= \int_{-\infty} ^\infty (\hat a_+ \hat a_- \psi_m) ^* \psi_n \dif x\\
        &= m \int_{-\infty} ^\infty \psi_m ^* \psi_n \dif x.
    \end{align*}
    Unless $m = n$, $\int_{-\infty} ^\infty \psi_m ^* \psi_n \dif x = 0$.
\end{pf}
Orthonormality means that we can again use 
\begin{equation}
    c_n = \int \psi_n ^* \Psi(x, 0) \dif x
\end{equation}
to determine the coefficients $\{ c_n \}$. $|c_n|^2$ is the probability that a measurement of the energy would return to the value $E_n$.\\
Using the definition (equation \ref{eq:ladder operators}), it is convenient to express $x$ and $\hat p$ in terms of the ladder operators:
\begin{align}
    x &= \sqrt{\frac{\hbar}{2m \omega}} (\hat a_+ + \hat a_-),\\
    \hat p &= i \sqrt{\frac{\hbar m \omega}{2}} (\hat a_+ - \hat a_-).
\end{align}

\subsection{Analytic Method}
    By introducing the dimensionless variable $\xi = \sqrt{\frac{m \omega}{\hbar}} x$, equation \eqref{eq:harmonic oscillator} becomes
    \begin{equation} \label{eq:harmonic oscillator dimensionless}
        \frac{\dif[2] \psi}{\dif \xi^2} = (\xi^2 - K) \psi,
    \end{equation}
    where
    \begin{equation} \label{eq:dimensionless energy}
        K = \frac{2E}{\hbar \omega}
    \end{equation}
    is the dimensionless energy.\\
    At very large $\xi$, the energy $K$ is negligible, and the solution is
    \begin{equation}
        \psi(\xi) \sim e^{-\xi^2/2},
    \end{equation}
    which suggests that the solution can be separated into two parts:
    \begin{equation} \label{eq:separated solution}
        \psi(\xi) = h(\xi) e^{-\xi^2/2}.
    \end{equation}
    Differentiating equation \eqref{eq:separated solution} twice, we get
    \begin{align}
        \frac{\dif \psi}{\dif \xi} &= \left( \frac{\dif h}{\dif \xi} - \xi h \right) e^{-\xi^2/2},\\
        \frac{\dif[2] \psi}{\dif \xi^2} &= \left( \frac{\dif[2] h}{\dif \xi^2} - 2\xi \frac{\dif h}{\dif \xi} + (\xi^2 - 1) h \right) e^{-\xi^2/2}.
    \end{align}
    Substituting these into equation \eqref{eq:harmonic oscillator dimensionless}, we get
    \begin{equation} \label{eq:harmonic oscillator dimensionless 2}
        \frac{\dif[2] h}{\dif \xi^2} - 2\xi \frac{\dif h}{\dif \xi} + (K - 1) h = 0.
    \end{equation}
    Rewrite $h(\xi)$ as a power series:
    \begin{equation}
        h(\xi) = \sum_{n=0}^\infty a_n \xi^n,
    \end{equation}
    whose derivatives are
    \begin{align}
        \frac{\dif h}{\dif \xi} &= \sum_{n=0}^\infty n a_n \xi^{n-1},\\
        \frac{\dif[2] h}{\dif \xi^2} &= \sum_{n=0}^\infty n(n-1) a_n \xi^{n-2}\\
        &= \sum_{n=0}^\infty (n+2)(n+1) a_{n+2} \xi^n.
    \end{align}
    Substituting these into equation \eqref{eq:harmonic oscillator dimensionless 2}, we get
    \begin{equation}
        \sum_{n=0}^\infty \left[ (n+2)(n+1) a_{n+2} - 2n a_n + (K - 1) a_n \right] \xi^n = 0.
    \end{equation}
    Since $\xi$ is arbitrary, the coefficient of each power of $\xi$ must vanish:
    \begin{equation}
        (n+2)(n+1) a_{n+2} - 2n a_n + (K - 1) a_n = 0,
    \end{equation}
    or
    \begin{equation} \label{eq:recursion relation}
        a_{n+2} = \frac{2n - K + 1}{(n+1)(n+2)} a_n.
    \end{equation}
    Starting with $a_0$, it generates all the even-numbered coefficients, 
    \begin{align*}
        a_2 &= \frac{1 - K}{2} a_0,\\
        a_4 &= \frac{5 - K}{12} a_2 = \frac{(5 - K)(1 - K)}{24} a_0,\\
        \cdots
    \end{align*}
    and starting with $a_1$, it generates all the odd-numbered coefficients,
    \begin{align*}
        a_3 &= \frac{3 - K}{6} a_1,\\
        a_5 &= \frac{7 - K}{20} a_3 = \frac{(7 - K)(3 - K)}{120} a_1,\\
        \cdots
    \end{align*}
    We write the complete solution as
    \begin{equation}
        h(\xi) = h_{even}(\xi) + h_{odd}(\xi),
    \end{equation}
    where
    \begin{equation*}
        h_{even}(\xi) = a_0 + a_2 \xi^2 + a_4 \xi^4 + \cdots 
    \end{equation*}
    is an even function of $\xi$, built on $a_0$, and
    \begin{equation*}
        h_{odd}(\xi) = a_1 \xi + a_3 \xi^3 + a_5 \xi^5 + \cdots
    \end{equation*}
    is an odd function of $\xi$, built on $a_1$.\\
    However, not all solutions so obtained are normalizable. For normalizable solutions, the power series must terminate. There must occur some highest $j$, say $n$. From equation \eqref{eq:recursion relation}, this means that the numerator must vanish for some $n$:
    \begin{equation}
        2n - K + 1 = 0 \quad \Rightarrow \quad K = 2n + 1,
    \end{equation}
    which leads to the quantization condition for the energy(equation \eqref{eq:dimensionless energy}):
    \begin{equation}
        E = \left( n + \frac{1}{2} \right) \hbar \omega, \quad n = 0, 1, 2, \cdots
    \end{equation}
    For allowed values of $K$, the recursion formula (equation \eqref{eq:recursion relation}) reads
    \begin{align}
        a_{j+2} &= \frac{2j - (2n + 1) + 1}{(j+1)(j+2)} a_j\\
        &= \frac{-2(n-j)}{(j+1)(j+2)} a_j.
    \end{align}
    If $n = 0$, there is only one coefficient, $a_0$. This gives the ground state wave function:
    \begin{equation*}
        h(\xi) = a_0,
    \end{equation*}
    and hence
    \begin{equation*}
        \psi_0(\xi) = a_0 e^{-\xi^2/2}.
    \end{equation*}
    If $n = 1$, we take $a_0 = 0$, which gives the first excited state:
    \begin{equation*}
        h(\xi) = a_1 \xi,
    \end{equation*}
    and hence
    \begin{equation*}
        \psi_1(\xi) = a_1 \xi e^{-\xi^2/2}.
    \end{equation*}
    If $n = 2$, $a_2 = -2 a_0$, which gives the second excited state:
    \begin{align*}
        h(\xi) &= a_0 + a_2 \xi^2\\
        &= a_0 (1 - 2 \xi^2),
    \end{align*}
    and hence
    \begin{equation*}
        \psi_2(\xi) = a_0 (1 - 2 \xi^2) e^{-\xi^2/2}.
    \end{equation*}
    In conclusion, the normalized stationary states of the harmonic oscillator are given by
    \begin{equation} \label{eq:harmonic oscillator stationary states}
        \psi_n(\xi) = \left( \frac{m \omega}{\pi \hbar} \right)^{1/4} \frac{1}{\sqrt{2^n n!}} H_n(\xi) e^{-\xi^2/2},
    \end{equation}
    where $H_n(\xi)$ is the $n$-th order Hermite polynomial.\\
    The quantum harmonic oscillator is strikingly different from its classical counterpart--not only are the energies quantized, but the position distributions have some bizarre features.
    For instance, the probability of finding the particle outside the  classically allowed region (i.e., $|x| > A$, where $A = \sqrt{\frac{2E}{m\omega^2}}$) is nonzero. 
    And in all odd states, the probability of finding the particle at $x = 0$ is zero, even though the classical oscillator spends most of its time near $x = 0$.
    \section{The Free Particle}
\end{document}