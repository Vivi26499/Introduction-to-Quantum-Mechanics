\documentclass[en, oneside]{vivi}

\ProjectInfos{Introduction to Quantum Mechanics}{PHY-512}{Spring, 2025}{Chapter2}{Due date: }{Vivi}[https://github.com/Vivi26499]{24S153073}

\begin{document}

\begin{prob}
    Prove the following three theorems:
    \begin{enumerate}[label=(\alph*)]
        \item For normalizable solutions, the separation constant $E$ must be real. 
        \textit{Hint: Write $E$ (in Equation 2.7) as $E_0 + i\Gamma$ (with $E_0$ and $\Gamma$ real), 
        and show that if Equation 1.20 is to hold for all $t$, $\Gamma$ must be zero.}
        \item The time-independent wave function $\psi(x)$ can always be taken to be real (unlike $\Psi(x, t)$, which is necessarily complex). 
        \textit{Hint: If $\psi(x)$ satisfies Equation 2.5, for a given $E$, so too does its complex conjugate, 
        and hence also the real linear combinations ($\psi + \psi^*$) and $i(\psi - \psi^*)$.}
        \item If $V(x)$ is an even function (that is, $V(-x) = V(x)$) then $\psi(x)$ can always be taken to be either even or odd. 
        \textit{Hint: If $\psi(x)$ satisfies Equation 2.5, for a given $E$, so too does $\psi(-x)$, 
        and hence also the even and odd linear combinations $\psi(x) \pm \psi(-x)$.}
    \end{enumerate}
\end{prob}

\begin{sol}
    \begin{enumerate}[label=(\alph*)]
        \item Suppose $E = E_0 + i\Gamma$ for some real $E_0$ and $\Gamma$. Then the time-dependent wave function $\Psi(x, t)$ can be written as
        \begin{align*}
            \Psi(x, t) &= \psi(x)e^{-iEt/\hbar}\\
            &= \psi(x)e^{-i(E_0 + i\Gamma)t/\hbar}\\
            &= \psi(x)e^{\Gamma t/\hbar}e^{-iE_0t/\hbar}.
        \end{align*}
        Thus,
        \begin{align*}
            \int_{-\infty}^{\infty}|\Psi(x, t)|^2 \dif x &= \int_{-\infty}^{\infty}|\psi(x)|^2e^{2\Gamma t/\hbar} \dif x\\
            &= e^{2\Gamma t/\hbar}\int_{-\infty}^{\infty}|\psi(x)|^2 \dif x,
        \end{align*}
        which varies with time, unless $\Gamma = 0$. Therefore, the separation constant $E$ must be real.
        \item If $\psi(x)$ satisfies $\hat H \psi = E\psi$, then its complex conjugate $\psi^*(x)$ also satisfies $\hat H \psi^* = E\psi^*$.\\
        If $\psi_1(x)$ and $\psi_2(x)$ are two solutions of $\hat H \psi = E\psi$, then any linear combination $\psi_3(x) = c_1\psi_1(x) + c_2\psi_2(x)$ is also a solution.\\
        Thus for any complex solution $\psi(x)$, we can construct two real solutions $\psi_1(x) = \frac{1}{2}(\psi(x) + \psi^*(x))$ and $\psi_2(x) = \frac{1}{2i}(\psi(x) - \psi^*(x))$.
        \item If $\psi(x)$ satisfies $ -\frac{\hbar^2}{2m}\frac{d^2\psi(x)}{ \dif x^2} + V(x)\psi(x) = E\psi(x)$, then
        \begin{align*}
            -\frac{\hbar^2}{2m}\frac{d^2\psi(-x)}{d(-x)^2} + V(-x)\psi(-x) &= -\frac{\hbar^2}{2m}\frac{d^2\psi(-x)}{ \dif x^2} + V(x)\psi(-x)\\
            &= E\psi(-x),
        \end{align*}
        which means $\psi(-x)$ is also a solution. 
        Thus we can construct two solutions $\psi_1(x) = \frac{1}{2}(\psi(x) + \psi(-x))$, which is even, and $\psi_2(x) = \frac{1}{2}(\psi(x) - \psi(-x))$, which is odd.
    \end{enumerate}
\end{sol}

\begin{prob}
    Show that $E$ must exceed the minimum value of $V(x)$, for every normalizable solution to the time-independent Schrödinger equation. 
    What is the classical analog to this statement? \textit{Hint: Rewrite Equation 2.5 in the form}
    \begin{equation*}
        \frac{d^2\psi}{ \dif x^2} = \frac{2m}{\hbar^2} [V(x) - E] \psi.
    \end{equation*}
    \textit{if $E < V_{\min}$, then $\psi$ and its second derivative always have the same sign—argue that such a function cannot be normalized.}
\end{prob}

\begin{sol}
    Rewrite time-independent Schrödinger equation as
    \begin{equation*}
        \frac{d^2\psi}{ \dif x^2} = \frac{2m}{\hbar^2} [V(x) - E] \psi.
    \end{equation*}
    If $E < V_{\min}$, then $V(x) - E > 0$ for all $x$. Thus $\psi$ and its second derivative always have the same sign, which means $\psi$ cannot be normalized.\\
    In classical mechanics, this statement is analogous that if the total energy of a particle is less than the minimum potential energy, the particle's kinetic energy is negative, then the particle cannot exist in the system.
\end{sol}

\begin{prob}
    Show that there is no acceptable solution to the (time-independent) Schrödinger equation for the infinite square well with $E = 0$ or $E < 0$. 
    (This is a special case of the general theorem in Problem 2.2, but this time do it by explicitly solving the Schrödinger equation, 
    and showing that you cannot satisfy the boundary conditions.)
\end{prob}

\begin{sol}
    When $E = 0$, the time-independent Schrödinger equation for the infinite square well becomes
    \begin{equation*}
        \frac{d^2\psi}{ \dif x^2} = -\frac{2mE}{\hbar^2} \psi = 0,
    \end{equation*}
    which leads to $\psi(x) = 0$, which is not normalizable.\\
    When $E < 0$, the time-independent Schrödinger equation for the infinite square well becomes
    \begin{equation*}
        \frac{d^2\psi}{ \dif x^2} = \kappa^2 \psi,
    \end{equation*}
    where $\kappa = \frac{\sqrt{-2mE}}{\hbar}$. The general solution to this equation is
    \begin{equation*}
        \psi(x) = A e^{\kappa x} + B e^{-\kappa x},
    \end{equation*}
    then the boundary conditions $\psi(0) = \psi(a) = 0$ lead to $A = B = 0$, which means $\psi(x) = 0$, which is not normalizable.
\end{sol}

\begin{prob}
    Calculate $\langle x \rangle$, $\langle x^2 \rangle$, $\langle p \rangle$, $\langle p^2 \rangle$, $\sigma_x$, and $\sigma_p$, 
    for the nth stationary state of the infinite square well. Check that the uncertainty principle is satisfied. 
    Which state comes closest to the uncertainty limit?
\end{prob}

\begin{sol}
    The expectation value of $x$ is
    \begin{align*}
        \langle x \rangle &= \int_{0}^{a}x|\psi_n(x)|^2 \dif x\\
        &= \int_{0}^{a}x\left(\sqrt{\frac{2}{a}}\sin\left(\frac{n\pi x}{a}\right)\right)^2 \dif x\\
        &= \frac{2}{a} \int_{0}^{a} x \sin^2\left(\frac{n\pi x}{a}\right) \dif x\\
        &= \frac{1}{a} \left[ \frac{1}{2}x^2 - \frac{a}{2n\pi} x \sin \frac{2 n \pi}{a} x - \frac{a^2}{4n^2\pi^2} \cos \frac{2 n \pi}{a} x \right]_0^a\\
        &= \frac{a}{2}.
    \end{align*}
    The expectation value of $x^2$ is
    \begin{align*}
        \langle x^2 \rangle &= \int_{0}^{a}x^2|\psi_n(x)|^2 \dif x\\
        &= \int_{0}^{a}x^2\left(\sqrt{\frac{2}{a}}\sin\left(\frac{n\pi x}{a}\right)\right)^2 \dif x\\
        &= \frac{2}{a} \int_{0}^{a} x^2 \sin^2\left(\frac{n\pi x}{a}\right) \dif x\\
        &= \frac{1}{a} \left[ \frac{1}{3}x^3 - \frac{a}{2n\pi} x^2 \sin \frac{2 n \pi}{a} x - \frac{a^2}{2 n^2 \pi^2} x \cos \frac{2 n \pi}{a} x + \frac{a^3}{4 n^3 \pi^3} \sin \frac{2 n \pi}{a} x \right]_0^a\\
        &= \frac{1}{a} \left( \frac{a^3}{3} - \frac{a^3}{2 n^2 \pi^2} \right)\\
        &= a^2 \left( \frac{1}{3} - \frac{1}{2 n^2 \pi^2} \right).
    \end{align*}
    The expectation value of $p$ is
    \begin{align*}
        \langle p \rangle &= m \frac{\dif \langle x \rangle}{\dif t}\\
        &= 0
    \end{align*}
    The expectation value of $p^2$ is
    \begin{align*}
        \langle p^2 \rangle &= \int_{0}^{a} \psi_n^*(x) \left( \frac{\hbar}{i} \frac{\dif}{\dif x} \right)^2 \psi_n \dif x\\
        &= -\hbar^2 \int_{0}^{a} \psi_n^*(x) \frac{\dif^2 \psi_n}{\dif x^2} \dif x\\
        &= -\hbar^2 (-\frac{2 m E_n}{\hbar^2}) \int_{0}^{a} |\psi_n(x)|^2 \dif x\\
        &= 2 m E_n\\
        &= \frac{n^2 \pi^2 \hbar^2}{a^2}.
    \end{align*}
    The standard deviation of $x$ is
    \begin{align*}
        \sigma_x &= \sqrt{\langle x^2 \rangle - \langle x \rangle^2}\\
        &= a \sqrt{\frac{1}{12} - \frac{1}{2 n^2 \pi^2}}.
    \end{align*}
    The standard deviation of $p$ is
    \begin{align*}
        \sigma_p &= \sqrt{\langle p^2 \rangle - \langle p \rangle^2}\\
        &= \frac{n \pi \hbar}{a}.
    \end{align*}
    The uncertainty principle is
    \begin{align*}
        \sigma_x \sigma_p &= a \sqrt{\frac{1}{12} - \frac{1}{2 n^2 \pi^2}} \cdot \frac{n \pi \hbar}{a}\\
        &= \frac{\hbar}{2} \sqrt{\frac{n^2 \pi^2}{3} - 2}\\
        &\geq \frac{\hbar}{2} \sqrt{\frac{\pi^2}{3} - 2}\\
        &\geq \frac{\hbar}{2}.
    \end{align*}
\end{sol}

\begin{prob}
    A particle in the infinite square well has as its initial wave function an even mixture of the first two stationary states:
    \begin{equation*}
        \Psi(x, 0) = A [\psi_1(x) + \psi_2(x)].
    \end{equation*}
    \begin{enumerate}[label=(\alph*)]
        \item Normalize $\Psi(x, 0)$. (That is, find $A$. This is very easy, if you exploit the orthonormality of $\psi_1$ and $\psi_2$. 
        Recall that, having normalized $\Psi$ at $t = 0$, you can rest assured that it stays normalized—if you doubt this, check it explicitly after doing part (b).)
        \item Find $\Psi(x, t)$ and $|\Psi(x, t)|^2$. Express the latter as a sinusoidal function of time, as in Example 2.1. 
        To simplify the result, let $\omega = \pi^2 \hbar / 2ma^2$.
        \item Compute $\langle x \rangle$. Notice that it oscillates in time. What is the angular frequency of the oscillation? 
        What is the amplitude of the oscillation? (If your amplitude is greater than $a/2$, go directly to jail.)
        \item Compute $\langle p \rangle$. (As Peter Lorre would say, “Do it ze kveek vay, Johnny!”)
        \item If you measured the energy of this particle, what values might you get, and what is the probability of getting each of them? 
        Find the expectation value of $H$. How does it compare with $E_1$ and $E_2$?
    \end{enumerate}
\end{prob}

\begin{sol}
    \begin{enumerate} [label=(\alph*)]
        \item \begin{align*}
            1 &= \int_{0}^{a}|\Psi(x, 0)|^2 \dif x\\
            &= A^2 \int_{0}^{a} [\psi_1(x) + \psi_2(x)]^* [\psi_1(x) + \psi_2(x)] \dif x\\
            &= A^2 \int_{0}^{a} \left[|\psi_1(x)|^2 + |\psi_2(x)|^2 + \psi_1^*(x)\psi_2(x) + \psi_2^*(x)\psi_1(x) \right] \dif x\\
            &= 2 A^2,
        \end{align*}
        so $A = \frac{1}{\sqrt{2}}$.
        \item \begin{align*}
            \Psi(x, t) &= \frac{1}{\sqrt{2}} \left[\psi_1(x) e^{-iE_1t/\hbar} + \psi_2(x) e^{-iE_2t/\hbar}\right]\\
            &= \frac{1}{\sqrt{2}} \left[\psi_1(x) e^{-i \omega t} + \psi_2(x) e^{-4i \omega t}\right]\\
            &= \frac{1}{\sqrt{2}} \left[\sqrt{\frac{2}{a}} \sin\left(\frac{\pi x}{a}\right) e^{-i \omega t} + \sqrt{\frac{2}{a}} \sin\left(\frac{2\pi x}{a}\right) e^{-4i \omega t}\right]\\
            &= \frac{1}{\sqrt{a}} e^{-i \omega t} \left[\sin\left(\frac{\pi x}{a}\right) + \sin\left(\frac{2\pi x}{a}\right) e^{-3i \omega t}\right].
        \end{align*}
        \begin{align*}
            |\Psi(x, t)|^2 &= \frac{1}{a} \left[\sin\left(\frac{\pi x}{a}\right) + \sin\left(\frac{2\pi x}{a}\right) e^{-3i \omega t}\right] \left[\sin\left(\frac{\pi x}{a}\right) + \sin\left(\frac{2\pi x}{a}\right) e^{-3i \omega t}\right]^*\\
            &= \frac{1}{a} \left[\sin\left(\frac{\pi x}{a}\right) + \sin\left(\frac{2\pi x}{a}\right) e^{-3i \omega t}\right] \left[\sin\left(\frac{\pi x}{a}\right) + \sin\left(\frac{2\pi x}{a}\right) e^{3i \omega t}\right]\\
            &= \frac{1}{a} \left[\sin^2\left(\frac{\pi x}{a}\right) + \sin^2\left(\frac{2\pi x}{a}\right) + \sin\left(\frac{\pi x}{a}\right)\sin\left(\frac{2\pi x}{a}\right) \left(e^{-3i \omega t} + e^{3i \omega t}\right)\right]\\
            &= \frac{1}{a} \left[\sin^2\left(\frac{\pi x}{a}\right) + \sin^2\left(\frac{2\pi x}{a}\right) + 2 \sin\left(\frac{\pi x}{a}\right)\sin\left(\frac{2\pi x}{a}\right) \cos(3\omega t)\right].
        \end{align*}
        \item \begin{align*}
            \langle x \rangle &= \int_{0}^{a}x|\Psi(x, t)|^2 \dif x\\
            &= \frac{1}{a} \int_{0}^{a}x \left[\sin^2\left(\frac{\pi x}{a}\right) + \sin^2\left(\frac{2\pi x}{a}\right) + 2 \sin\left(\frac{\pi x}{a}\right)\sin\left(\frac{2\pi x}{a}\right) \cos(3\omega t)\right] \dif x\\
            &= \frac{1}{a} \int_{0}^{a}x \left[\sin^2\left(\frac{\pi x}{a}\right) + \sin^2\left(\frac{2\pi x}{a}\right) \dif x \right] + \frac{2}{a} \cos(3\omega t) \int_{0}^{a}x \sin\left(\frac{\pi x}{a}\right)\sin\left(\frac{2\pi x}{a}\right) \dif x\\
            &= \frac{1}{a} \left[\frac{a^2}{4} + \frac{a^2}{4}\right] + \frac{1}{a} \cos(3\omega t) \int_{0}^{a}x \left[ \cos\left(\frac{\pi x}{a}\right) - \cos\left(\frac{3\pi x}{a}\right) \right] \dif x\\
            &= \frac{a}{2} + \frac{1}{a} \cos(3\omega t) \left[ \frac{a}{\pi} x \sin\left(\frac{\pi x}{a}\right) + \frac{a^2}{\pi^2} \cos\left(\frac{\pi x}{a}\right) - \frac{a}{3\pi} x \sin\left(\frac{3\pi x}{a}\right) - \frac{a^2}{9\pi^2} \cos\left(\frac{3\pi x}{a}\right) \right]_0^a\\
            &= \frac{a}{2} + \frac{1}{a} \cos(3\omega t) \left[ - \frac{a^2}{\pi^2} - \frac{a^2}{\pi^2} + \frac{a^2}{9\pi^2} + \frac{a^2}{9\pi^2} \right]\\
            &= \frac{a}{2} - \frac{16}{9\pi^2} a \cos(3\omega t)\\
            &= \frac{a}{2} \left[ 1 - \frac{32}{9\pi^2} \cos(3\omega t) \right],
        \end{align*}
        where the angular frequency of the oscillation is $3\omega = \frac{3\pi^2 \hbar}{2ma^2}$ and the amplitude of the oscillation is $\frac{16 a}{9\pi^2} \approx 0.18 a$.
        \item \begin{align*}
            \langle p \rangle &= m \frac{\dif \langle x \rangle}{\dif t}\\
            &= m \frac{\dif}{\dif t} \left[ \frac{a}{2} \left( 1 - \frac{32}{9\pi^2} \cos(3\omega t) \right) \right]\\
            &= \frac{16 m a}{9\pi^2} 3\omega \sin(3\omega t)\\
            &= \frac{8 \hbar}{3a} \sin(3\omega t).
        \end{align*}
        \item The possible values of energy are $E_1 = \frac{\pi^2 \hbar^2}{2ma^2}$ and $E_2 = \frac{2\pi^2 \hbar^2}{ma^2}$, with the probability of getting each of them being $\frac{1}{2}$. The expectation value of $H$ is
        \begin{align*}
            \langle H \rangle &= \frac{1}{2} E_1 + \frac{1}{2} E_2\\
            &= \frac{5\pi^2 \hbar^2}{4ma^2}.
        \end{align*}
    \end{enumerate}
\end{sol}

\begin{prob}
    Although the overall phase constant of the wave function is of no physical significance (it cancels out whenever you calculate a measurable quantity), the relative phase of the coefficients in Equation 2.17 does matter. For example, suppose we change the relative phase of $\psi_1$ and $\psi_2$ in Problem 2.5:
    \begin{equation*}
        \Psi(x, 0) = A \left[ \psi_1(x) + e^{i\phi}\psi_2(x) \right],
    \end{equation*}
    where $\phi$ is some constant. Find $\Psi(x, t)$, $|\Psi(x, t)|^2$, and $\langle x \rangle$, and compare your results with what you got before. Study the special cases $\phi = \pi/2$ and $\phi = \pi$. (For a graphical exploration of this problem see the applet in footnote 9 of this chapter.)
\end{prob}

\begin{sol}
    \begin{align*}
        1 &= \int_{0}^{a}|\Psi(x, 0)|^2 \dif x\\
        &= A^2 \int_{0}^{a} \left[|\psi_1(x)|^2 + |\psi_2(x)|^2 + e^{i\phi}\psi_1^*(x)\psi_2(x) + e^{-i\phi}\psi_2^*(x)\psi_1(x) \right] \dif x\\
        &= 2 A^2,
    \end{align*}
    so $A = \frac{1}{\sqrt{2}}$.
    \begin{align*}
        \Psi(x, t) &= \frac{1}{\sqrt{2}} \left[\psi_1(x) e^{-iE_1t/\hbar} + e^{i\phi}\psi_2(x) e^{-iE_2t/\hbar}\right]\\
        &= \frac{1}{\sqrt{2}} \left[\psi_1(x) e^{-i \omega t} + e^{i\phi}\psi_2(x) e^{-4i \omega t}\right]\\
        &= \frac{1}{\sqrt{2}} \left[\psi_1(x) e^{-i \omega t} + e^{i\phi}\psi_2(x) e^{-4i \omega t}\right]\\
        &= \frac{1}{\sqrt{2}} \left[\sqrt{\frac{2}{a}} \sin\left(\frac{\pi x}{a}\right) e^{-i \omega t} + e^{i\phi}\sqrt{\frac{2}{a}} \sin\left(\frac{2\pi x}{a}\right) e^{-4i \omega t}\right]\\
        &= \frac{1}{\sqrt{a}} e^{-i \omega t} \left[\sin\left(\frac{\pi x}{a}\right) + \sin\left(\frac{2\pi x}{a}\right) e^{i\phi} e^{-3i \omega t}\right].
    \end{align*}
    \begin{align*}
        |\Psi(x, t)|^2 &= \frac{1}{a} \left[\sin^2\left(\frac{\pi x}{a}\right) + \sin^2\left(\frac{2\pi x}{a}\right) + \sin\left(\frac{\pi x}{a}\right)\sin\left(\frac{2\pi x}{a}\right) \left( e^{i\phi} e^{-3i \omega t} + e^{-i\phi} e^{3i \omega t} \right)\right]\\
        &= \frac{1}{a} \left[\sin^2\left(\frac{\pi x}{a}\right) + \sin^2\left(\frac{2\pi x}{a}\right) + 2 \sin\left(\frac{\pi x}{a}\right)\sin\left(\frac{2\pi x}{a}\right) \cos(3\omega t - \phi)\right].
    \end{align*}
    Then $\langle x \rangle = \frac{a}{2} \left[ 1 - \frac{32}{9\pi^2} \cos(3\omega t - \phi) \right]$.\\
    When $\phi = \pi/2$, $\langle x \rangle = \frac{a}{2} \left[ 1 + \frac{32}{9\pi^2} \sin(3\omega t) \right]$,\\
    When $\phi = \pi$, $\langle x \rangle = \frac{a}{2} \left[ 1 + \frac{32}{9\pi^2} \cos(3\omega t) \right]$.
\end{sol}

\begin{prob}
    A particle in the infinite square well has the initial wave function
    \begin{equation*}
        \Psi(x, 0) =
        \begin{cases}
            Ax, & 0 \leq x \leq a/2, \\
            A(a - x), & a/2 \leq x \leq a.
        \end{cases}
    \end{equation*}
    \begin{enumerate}[label=(\alph*)]
        \item Sketch $\Psi(x, 0)$, and determine the constant $A$.
        \item Find $\Psi(x, t)$.
        \item What is the probability that a measurement of the energy would yield the value $E_1$?
        \item Find the expectation value of the energy, using Equation 2.21.
    \end{enumerate}
\end{prob}

\begin{sol}
    \begin{enumerate}[label=(\alph*)]
        \item \begin{align*}
            1 &= \int_{0}^{a}|\Psi(x, 0)|^2 \dif x\\
            &= A^2 \left[ \int_{0}^{a/2}x^2 \dif x + \int_{a/2}^{a}(a - x)^2 \dif x \right]\\
            &= A^2 \left[ \frac{x^3}{3} \bigg|_{0}^{a/2} - \frac{(a - x)^3}{3} \bigg|_{a/2}^{a} \right]\\
            &= A^2 \left[ \frac{a^3}{24} + \frac{a^3}{24} \right]\\
            &= \frac{A^2 a^3}{12},
        \end{align*}
        so $A = \frac{2 \sqrt{3}}{a\sqrt{a}}$.
        \item \begin{align*}
            c_n &= \int_{0}^{a} \Psi(x, 0) \psi_n^*(x) \dif x\\
            &= A \sqrt{\frac{2}{a}} \left[\int_{0}^{a/2} x \sin\left(\frac{n\pi x}{a}\right) \dif x + \int_{a/2}^{a} (a - x) \sin\left(\frac{n\pi x}{a}\right) \dif x\right]\\
            &= A \sqrt{\frac{2}{a}} \frac{a}{n\pi} \left[ -x \cos \frac{n\pi x}{a} \bigg|_{0}^{a/2} + \int_{0}^{a/2} \cos \frac{n\pi x}{a} \dif x - (a - x) \cos \frac{n\pi x}{a} \bigg|_{a/2}^{a} - \int_{a/2}^{a} \cos \frac{n\pi x}{a} \dif x \right]\\
            &= A \sqrt{\frac{2}{a}} \frac{a}{n\pi} \left[\frac{a}{n\pi} \sin \frac{n\pi x}{a} \bigg|_{0}^{a/2} - \frac{a}{n\pi} \sin \frac{n\pi x}{a} \bigg|_{a/2}^{a} \right]\\
            &= A \sqrt{\frac{2}{a}} \frac{a}{n\pi} \left[\frac{a}{n\pi} \sin \frac{n\pi}{2} + \frac{a}{n\pi} \sin \frac{n\pi}{2} \right]\\
            &= \frac{4 \sqrt{6}}{n^2 \pi^2} \sin \frac{n\pi}{2}\\
            &= \begin{cases}
                \frac{4 \sqrt{6}}{n^2 \pi^2} (-1)^{(n - 1)/2}, & n = 1, 3, 5, \cdots,\\
                0, & n = 2, 4, 6, \cdots.
            \end{cases}
        \end{align*}
        So \begin{align*}
            \Psi(x, t) &= \sum_{n = 1, 3, 5, \cdots} c_n \psi_n(x) e^{-iE_n t/\hbar}\\
            &= \sum_{n = 1, 3, 5, \cdots} \frac{4 \sqrt{6}}{n^2 \pi^2} (-1)^{(n - 1)/2} \sqrt{\frac{2}{a}} \sin\left(\frac{n\pi x}{a}\right) e^{-iE_n t/\hbar}\\
            &= \frac{4 \sqrt{6}}{\pi^2} \sqrt{\frac{2}{a}} \sum_{n = 1, 3, 5, \cdots} \frac{(-1)^{(n - 1)/2}}{n^2} \sin\left(\frac{n\pi x}{a}\right) e^{-iE_n t/\hbar},
        \end{align*}
        where $E_n = \frac{n^2 \pi^2 \hbar^2}{2ma^2}$.
        \item The probability that a measurement of the energy would yield the value $E_1$ is
        \begin{align*}
            P(E_1) &= |c_1|^2\\
            &= \left(\frac{4 \sqrt{6}}{\pi^2}\right)^2\\
            &= \frac{96}{\pi^4} \approx 0.9855
        \end{align*}
        \item The expectation value of the energy is
        \begin{align*}
            \langle H \rangle &= \sum_{n = 1, 3, 5, \cdots} |c_n|^2 E_n\\
            &= \sum_{n = 1, 3, 5, \cdots} \left( \frac{4 \sqrt{6}}{n^2 \pi^2} \right)^2 \frac{n^2 \pi^2 \hbar^2}{2ma^2}\\
            &= \sum_{n = 1, 3, 5, \cdots} \frac{48 \hbar^2}{n^2 \pi^2 m a^2}\\
            &= \frac{48 \hbar^2}{\pi^2 m a^2} \sum_{n = 1, 3, 5, \cdots} \frac{1}{n^2}\\
            &= \frac{48 \hbar^2}{\pi^2 m a^2} \frac{\pi^2}{8}\\
            &= \frac{6 \hbar^2}{m a^2}.
        \end{align*}
    \end{enumerate}
\end{sol}

\begin{prob}
    A particle of mass $m$ in the infinite square well (of width $a$) starts out in the state
    \begin{equation*}
        \Psi(x, 0) =
        \begin{cases}
            A, & 0 \leq x \leq a/2, \\
            0, & a/2 < x \leq a,
        \end{cases}
    \end{equation*}
    for some constant $A$, so it is (at $t = 0$) equally likely to be found at any point in the left half of the well. What is the probability that a measurement of the energy (at some later time $t$) would yield the value $\pi^2 \hbar^2 / 2ma^2$?
\end{prob}

\begin{sol}
    \begin{align*}
        1 &= \int_{0}^{a}|\Psi(x, 0)|^2 \dif x\\
        &= A^2 \left[ \int_{0}^{a/2} \dif x \right]\\
        &= \frac{A^2 a}{2},
    \end{align*}
    so $A = \sqrt{\frac{2}{a}}$.
    \begin{align*}
        c_1 &= \int_{0}^{a} \Psi(x, 0) \psi_1^*(x) \dif x\\
        &= A \sqrt{\frac{2}{a}} \int_{0}^{a/2} \sin\left(\frac{\pi x}{a}\right) \dif x\\
        &= A \sqrt{\frac{2}{a}} \frac{a}{\pi} \left[ -\cos\left(\frac{\pi x}{a}\right) \bigg|_{0}^{a/2} \right]\\
        &= \frac{2}{\pi}.
    \end{align*}
    The probability that a measurement of the energy would yield the value $\pi^2 \hbar^2 / 2ma^2 = E_1$ is
    \begin{align*}
        P(E_1) &= |c_1|^2\\
        &= \left(\frac{2}{\pi}\right)^2\\
        &= \frac{4}{\pi^2} \approx 0.4053.
    \end{align*}
\end{sol}

\begin{prob}
    For the wave function in Example 2.2, find the expectation value of $H$, at time $t = 0$, the “old fashioned” way:
    \begin{equation*}
        \langle H \rangle = \int \Psi(x, 0)^* \hat{H} \Psi(x, 0) \dif x.
    \end{equation*}
    Compare the result we got in Example 2.3. Note: Because $\langle H \rangle$ is independent of time, there is no loss of generality in using $t = 0$.
\end{prob}

\begin{sol}
    \begin{align*}
        \hat H \Psi(x, 0) &= \frac{-\hbar^2}{2m} \frac{\dif^2}{\dif x^2} \Psi(x, 0)\\
        &= \frac{-\hbar^2}{2m} \frac{\dif^2}{\dif x^2} \left[ Ax(a - x) \right]\\
        &= \frac{-\hbar^2}{2m} (-2A)\\
        &= \frac{\hbar^2 A}{m}.
    \end{align*}
    \begin{align*}
        \langle H \rangle &= \int \Psi(x, 0)^* \hat{H} \Psi(x, 0) \dif x\\
        &= \frac{\hbar^2 A^2}{m} \int_{0}^{a} x(a - x) \dif x\\
        &= \frac{\hbar^2 A^2}{m} \left[ \frac{ax^2}{2} - \frac{x^3}{3} \right]_0^a\\
        &= \frac{\hbar^2 A^2 a^3}{6m}\\
        &= \frac{5 \hbar^2}{m a^2}.
    \end{align*}
\end{sol}

\begin{prob}
    \begin{enumerate}[label=(\alph*)]
        \item Construct $\psi_2(x)$.
        \item Sketch $\psi_0$, $\psi_1$, and $\psi_2$.
        \item Check the orthogonality of $\psi_0$, $\psi_1$, and $\psi_2$, by explicit integration. 
        \textit{Hint: If you exploit the even-ness and odd-ness of the functions, there is really only one integral left to do.}
    \end{enumerate}
\end{prob}

\begin{sol}
    \begin{enumerate}[label=(\alph*)]
        \item \begin{equation*}
            \psi_0 = \left( \frac{m \omega}{\pi \hbar} \right)^{1/4} e^{-\frac{m \omega}{2\hbar}x^2},
        \end{equation*}
        \begin{align*}
            \hat a_+ \psi_0 &= \frac{1}{\sqrt{2\hbar m \omega}} \left( -i \hat p + m \omega \hat x \right) \psi_0\\
            &= \frac{1}{\sqrt{2\hbar m \omega}} \left( \frac{m \omega}{\pi \hbar} \right)^{1/4} \left( - \hbar \frac{\dif}{\dif x} + m \omega x \right) e^{-\frac{m \omega}{2\hbar}x^2}\\
            &= \frac{1}{\sqrt{2\hbar m \omega}} \left( \frac{m \omega}{\pi \hbar} \right)^{1/4} \left[ - \hbar \left( -\frac{m \omega}{\hbar} x \right) + m \omega x \right] e^{-\frac{m \omega}{2\hbar}x^2}\\
            &= \frac{1}{\sqrt{2\hbar m \omega}} \left( \frac{m \omega}{\pi \hbar} \right)^{1/4} 2 m \omega x e^{-\frac{m \omega}{2\hbar}x^2}.
        \end{align*}
        \begin{align*}
            (\hat a_+)^2 \psi_0 &= \frac{1}{\sqrt{2\hbar m \omega}} \left( -i \hat p + m \omega \hat x \right) \left( \hat a_+ \psi_0 \right)\\
            &= \frac{1}{2 \hbar m \omega} \left( \frac{m \omega}{\pi \hbar} \right)^{1/4} 2 m \omega \left( - \hbar \frac{\dif}{\dif x} + m \omega x \right) x e^{-\frac{m \omega}{2\hbar}x^2}\\
            &= \frac{1}{2 \hbar m \omega} \left( \frac{m \omega}{\pi \hbar} \right)^{1/4} 2 m \omega \left[ - \hbar \left( 1 -\frac{m \omega}{\hbar} x^2 \right) + m \omega x^2 \right] e^{-\frac{m \omega}{2\hbar}x^2}\\
            &= \left( \frac{m \omega}{\pi \hbar} \right)^{1/4} \left[\frac{2 m \omega}{\hbar} x^2 - 1\right] e^{-\frac{m \omega}{2\hbar}x^2}.
        \end{align*}
        Therefore,
        \begin{align*}
            \psi_2 &= \frac{1}{\sqrt{2}} \left( \hat a_+ \right)^2 \psi_0\\
            &= \frac{1}{\sqrt{2}} \left( \frac{m \omega}{\pi \hbar} \right)^{1/4} \left[\frac{2 m \omega}{\hbar} x^2 - 1\right] e^{-\frac{m \omega}{2\hbar}x^2}.
        \end{align*}
        \item 
        \item As $\psi_0$ and $\psi_2$ are even and $\psi_1$ is odd, the only integral left to do is
        \begin{align*}
            \int_{-\infty}^{\infty} \psi_0^* \psi_2 \dif x &= \int_{-\infty}^{\infty} \left( \frac{m \omega}{\pi \hbar} \right)^{1/4} e^{-\frac{m \omega}{2\hbar}x^2} \frac{1}{\sqrt{2}} \left( \frac{m \omega}{\pi \hbar} \right)^{1/4} \left[\frac{2 m \omega}{\hbar} x^2 - 1\right] e^{-\frac{m \omega}{2\hbar}x^2} \dif x\\
            &= \frac{1}{\sqrt{2}} \left( \frac{m \omega}{\pi \hbar} \right)^{1/2} \int_{-\infty}^{\infty} \left[\frac{2 m \omega}{\hbar} x^2 - 1\right] e^{-\frac{m \omega}{\hbar}x^2} \dif x\\
            &= \frac{1}{\sqrt{2}} \left( \frac{m \omega}{\pi \hbar} \right)^{1/2} \left[ \frac{2 m \omega}{\hbar} \int_{-\infty}^{\infty} x^2 e^{-\frac{m \omega}{\hbar}x^2} \dif x - \int_{-\infty}^{\infty} e^{-\frac{m \omega}{\hbar}x^2} \dif x \right]\\
            &= \frac{1}{\sqrt{2}} \left( \frac{m \omega}{\pi \hbar} \right)^{1/2} \left[ \frac{2 m \omega}{\hbar} \frac{\hbar}{2 m \omega} \sqrt{\frac{\pi \hbar}{m \omega}} - \sqrt{\frac{\pi \hbar}{m \omega}} \right]\\
            &= 0.
        \end{align*}
    \end{enumerate}
\end{sol}
\begin{prob}
    \bigskip
    \begin{enumerate}[label=(\alph*)]
        \item Compute $\langle x \rangle$, $\langle p \rangle$, $\langle x^2 \rangle$, and $\langle p^2 \rangle$, for the states $\psi_0$ (Equation 2.60) and $\psi_1$ (Equation 2.63), by explicit integration.
        \item Check the uncertainty principle for these states.
        \item Compute $\langle T \rangle$ and $\langle V \rangle$ for these states. 
        \textit{(No new integration allowed!) Is their sum what you would expect?}
    \end{enumerate}
\end{prob}

\begin{prob}
    Find $\langle x \rangle$, $\langle p \rangle$, $\langle x^2 \rangle$, $\langle p^2 \rangle$, and $\langle T \rangle$, for the $n$th stationary state of the harmonic oscillator, using the method of Example 2.5. 
    Check that the uncertainty principle is satisfied.
\end{prob}

\begin{sol}
    The expectation value of $x$ is
    \begin{align*}
        \langle x \rangle &= \sqrt{\frac{\hbar}{2m\omega}} \int_{-\infty}^{\infty} \psi_n^* (\hat a_+ + \hat a_-) \psi_n \dif x\\
        &= \sqrt{\frac{\hbar}{2m\omega}} \left[ \sqrt{n + 1} \int_{-\infty}^{\infty} \psi_n^* \psi_{n+1} + \sqrt{n} \int_{-\infty}^{\infty} \psi_n^* \psi_{n-1} \dif x \right]\\
        &= 0.
    \end{align*}
    Thus the expectation value of $p$ is 
    \begin{align*}
        \langle p \rangle &= \frac{\dif}{\dif t} \langle x \rangle\\
        &= 0.
    \end{align*}
    The expectation value of $x^2$ is
    \begin{align*}
        \langle x^2 \rangle &= \frac{\hbar}{2m\omega} \int_{-\infty}^{\infty} \psi_n^* (\hat a_+ + \hat a_-)^2 \psi_n \dif x\\
        &= \frac{\hbar}{2m\omega} \int_{-\infty}^{\infty} \psi_n^* (\hat a_+^2 + \hat a_-^2 + \hat a_+ \hat a_- + \hat a_- \hat a_+) \psi_n \dif x\\
        &= \frac{\hbar}{2m\omega} \left[ \sqrt{n+1} \int_{-\infty}^{\infty} \psi_n^* \hat a_+ \psi_{n+1} + \sqrt{n} \int_{-\infty}^{\infty} \psi_n^* \hat a_- \psi_{n-1} \dif x + (n+n+1)\int_{-\infty}^{\infty} \psi_n^* \psi_n \dif x \right]\\
        &= \frac{\hbar}{2m\omega} \left[ \sqrt{n+1} \sqrt{n+2} \int_{-\infty}^{\infty} \psi_n^* \psi_{n+2} + n \sqrt{n-1} \int_{-\infty}^{\infty} \psi_n^* \psi_{n-2} \dif x + 2n+1 \right]\\
        &= \left( n+\frac{1}{2} \right) \frac{\hbar}{m\omega}.
    \end{align*}
    The expectation value of $p^2$ is
    \begin{align*}
        \langle p^2 \rangle &= - \frac{\hbar m \omega}{2} \int_{-\infty}^{\infty} \psi_n^* (\hat a_+ - \hat a_-)^2 \psi_n \dif x\\
        &= - \frac{\hbar m \omega}{2} \int_{-\infty}^{\infty} \psi_n^* (\hat a_+^2 + \hat a_-^2 - \hat a_+ \hat a_- - \hat a_- \hat a_+) \psi_n \dif x\\
        &= - \frac{\hbar m \omega}{2} \left[ \sqrt{n+1} \int_{-\infty}^{\infty} \psi_n^* \hat a_+ \psi_{n+1} + \sqrt{n} \int_{-\infty}^{\infty} \psi_n^* \hat a_- \psi_{n-1} \dif x - (n+n+1)\int_{-\infty}^{\infty} \psi_n^* \psi_n \dif x \right]\\
        &= - \frac{\hbar m \omega}{2} \left[ \sqrt{n+1} \sqrt{n+2} \int_{-\infty}^{\infty} \psi_n^* \psi_{n+2} + n \sqrt{n-1} \int_{-\infty}^{\infty} \psi_n^* \psi_{n-2} \dif x - (2n+1) \right]\\
        &= \left( n+\frac{1}{2} \right) \hbar m \omega.
    \end{align*}
    The expectation value of $T$ is
    \begin{align*}
        \langle T \rangle &= \frac{1}{2m} \langle p^2 \rangle\\
        &= \frac{1}{2} \left( n+\frac{1}{2} \right) \hbar \omega.
    \end{align*}
    The standard deviation of $x$ is
    \begin{align*}
        \sigma_x &= \sqrt{\langle x^2 \rangle - \langle x \rangle^2}\\
        &= \sqrt{\left( n+\frac{1}{2} \right) \frac{\hbar}{m\omega}}.
    \end{align*}
    The standard deviation of $p$ is
    \begin{align*}
        \sigma_p &= \sqrt{\langle p^2 \rangle - \langle p \rangle^2}\\
        &= \sqrt{\left( n+\frac{1}{2} \right) \hbar m \omega}.
    \end{align*}
    The uncertainty principle is
    \begin{align*}
        \sigma_x \sigma_p &= \sqrt{\left( n+\frac{1}{2} \right) \frac{\hbar}{m\omega}} \sqrt{\left( n+\frac{1}{2} \right) \hbar m \omega}\\
        &= \left( n+\frac{1}{2} \right) \hbar \geq \frac{\hbar}{2}.
    \end{align*}
\end{sol}

\begin{prob}
    A particle in the harmonic oscillator potential starts out in the state
    \begin{equation*}
        \Psi(x, 0) = A \left[ 3\psi_0(x) + 4\psi_1(x) \right].
    \end{equation*}
    \begin{enumerate}[label=(\alph*)]
        \item Find $A$.
        \item Construct $\Psi(x, t)$ and $|\Psi(x, t)|^2$. Don’t get too excited if $|\Psi(x, t)|^2$ oscillates at exactly the classical frequency; what would it have been had I specified $\psi_2(x)$, instead of $\psi_1(x)$?
        \item Find $\langle x \rangle$ and $\langle p \rangle$. Check that Ehrenfest’s theorem (Equation 1.38) holds, for this wave function.
        \item If you measured the energy of this particle, what values might you get, and with what probabilities?
    \end{enumerate}
\end{prob}

\begin{sol}
    \begin{enumerate}[label=(\alph*)]
        \item \begin{align*}
            1 &= \int_{-\infty}^{\infty} |\Psi(x, 0)|^2 \dif x\\
            &= |A|^2 ( 9 + 16 ),
        \end{align*}
        so $A = \frac{1}{5}$.
        \item \begin{align*}
            \Psi(x, t) &= \frac{1}{5} \left[ 3\psi_0(x) e^{-iE_0t/\hbar} + 4\psi_1(x) e^{-iE_1t/\hbar} \right]\\
            &= \frac{1}{5} \left[ 3\psi_0(x) e^{-\frac{1}{2}i\omega t} + 4\psi_1(x) e^{-\frac{3}{2}i\omega t} \right].
        \end{align*}
        \begin{align*}
            |\Psi(x, t)|^2 &= \frac{1}{25} \left[ 9|\psi_0(x)|^2 + 16|\psi_1(x)|^2 + 12 \psi_0(x) \psi_1(x) (e^{i\omega t} + e^{-i\omega t}) \right]\\
            &= \frac{1}{25} \left[ 9|\psi_0(x)|^2 + 16|\psi_1(x)|^2 + 24 \psi_0(x) \psi_1(x) \cos(\omega t) \right].
        \end{align*}
        \item \begin{align*}
            \langle x \rangle &= \int_{-\infty}^{\infty} \Psi(x, 0)^* x \Psi(x, 0) \dif x\\
            &= \frac{1}{25} \left[ 9 \int_{-\infty}^{\infty} \psi_0 x \psi_0 \dif x + 16 \int_{-\infty}^{\infty} \psi_1 x \psi_1 \dif x + 24 \cos(\omega t) \int_{-\infty}^{\infty} \psi_0 x \psi_1 \dif x \right]\\
            &= \frac{24}{25} \cos(\omega t) \int_{-\infty}^{\infty} \psi_0 x \psi_1 \dif x\\
            &= \frac{24}{25} \cos(\omega t) \sqrt{\frac{m \omega}{\pi \hbar}} \sqrt{\frac{2 m \omega}{\hbar}} \int_{-\infty}^\infty x e^{-\frac{m\omega}{2\hbar}x^2} x e^{-\frac{m\omega}{2\hbar}x^2} \dif x\\
            &= \frac{24}{25} \cos(\omega t) \sqrt{\frac{1}{2\pi}} \int_{-\infty}^\infty e^{-\frac{m \omega}{\hbar}x^2} \dif x\\
            &= \frac{24}{25} \sqrt{\frac{\hbar}{2 m \omega}} \cos(\omega t).
        \end{align*}
        Then,
        \begin{align*}
            \langle p \rangle &= m \frac{\dif}{\dif t} \langle x \rangle\\
            &= -\frac{24}{25} \sqrt{\frac{\hbar m \omega}{2}} \sin(\omega t).
        \end{align*}
        We have
        \begin{equation*}
            \frac{\dif \langle p \rangle}{\dif t} = -\frac{24}{25} \sqrt{\frac{\hbar m \omega}{2}} \omega \cos(\omega t),
        \end{equation*}
        and
        \begin{equation*}
            \frac{\dif V}{\dif x} = m \omega^2 x.
        \end{equation*}
        Therefore,
        \begin{align*}
            - \langle \frac{\dif V}{\dif x} \rangle &= - m \omega^2 \langle x \rangle\\
            &= - \frac{24}{25} \sqrt{\frac{\hbar m \omega}{2}} \omega \cos(\omega t)\\
            &= \frac{\dif \langle p \rangle}{\dif t},
        \end{align*}
        so Ehrenfest's theorem holds.
        \item We can get the energy $E_0 = \frac{\hbar \omega}{2}$ with probability $\frac{9}{25}$ and the energy $E_1 = \frac{3\hbar \omega}{2}$ with probability $\frac{16}{25}$.
    \end{enumerate}
\end{sol}
\end{document}
