\documentclass[en, oneside]{vivi}

\ProjectInfos{Introduction to Quantum Mechanics}{PHY-512}{Spring, 2025}{Chapter1}{Due date: }{Vivi}[https://github.com/Vivi26499]{24S153073}

\begin{document}

\begin{prob}
    For the distribution of ages in the example in Section 1.3.1:
    \begin{enumerate}[label=(\alph*)]
        \item Compute $\langle j^2 \rangle$ and $\langle j \rangle^2$.
        \item Determine $\Delta j$ for each $j$, and use Equation 1.11 to compute the standard deviation.
        \item Use your results in (a) and (b) to check Equation 1.12.
    \end{enumerate}
\end{prob}

\begin{prob}
    \begin{enumerate}[label=(\alph*)]
        \item Find the standard deviation of the distribution in Example 1.2.
        \item What is the probability that a photograph, selected at random, would show a distance $x$ more than one standard deviation away from the average?
    \end{enumerate}
\end{prob}

\begin{prob}
    Consider the gaussian distribution
    \begin{equation*}
        \rho(x) = A e^{-\lambda (x - a)^2},
    \end{equation*}
    where $A$, $a$, and $\lambda$ are positive real constants. (The necessary integrals are inside the back cover.)
    \begin{enumerate}[label=(\alph*)]
        \item Use Equation 1.16 to determine $A$.
        \item Find $\langle x \rangle$, $\langle x^2 \rangle$, and $\sigma$.
        \item Sketch the graph of $\rho(x)$.
    \end{enumerate}
\end{prob}

\begin{prob}
    At time $t = 0$, a particle is presented by the wave function
    \begin{equation*}
        \Psi (x, 0) = \begin{cases}
            A (x/a), & 0 \leq x \leq a,\\
            A (b - x)/(b - a), & a \leq x \leq b,\\
            0, & \text{otherwise},
        \end{cases}
    \end{equation*}
    where $A$, $a$, and $b$ are positive constants.
    \begin{enumerate}[label=(\alph*)]
        \item Normalize $\Psi$ (that is, find A, in terms of $a$ and $b$).
        \item Sketch $\Psi (x, 0)$, as a function of $x$.
        \item Where is the particle most likely to be found at $t = 0$?
        \item What is the probability of finding the particle to the left of $a$? Check your result in the limiting cases $b = a$ and $b = 2a$.
        \item What is the expectation value of $x$.
    \end{enumerate}
\end{prob}

\begin{sol}
    \begin{enumerate}[label=(\alph*)]
        \item \begin{align*}
            1 &= \int_{-\infty}^{\infty} |\Psi (x, 0)|^2 \dif x\\
            &= \int_{0}^{a} |A (x/a)|^2 \dif x + \int_{a}^{b} |A (b - x)/(b - a)|^2 \dif x\\
            &= \frac{A^2}{a^2} \int_{0}^{a} x^2 \dif x + \frac{A^2}{(b - a)^2} \int_{a}^{b} (b - x)^2 \dif x\\
            &= A^2 \left( \frac{a}{3} + \frac{b - a}{3} \right)\\
            &= A^2 \frac{b}{3}.
        \end{align*}
        Thus, we have $A = \sqrt{\frac{3}{b}}$.
        \item The sketch of $\Psi (x, 0)$ is shown below.
        \begin{center}
            \begin{tikzpicture}
            \draw[->] (-0.5, 0) -- (4.5, 0) node[right] {$x$};
            \draw[->] (0, -0.5) -- (0, 2.5) node[above] {$\Psi(x, 0)$};
            \draw[scale=1, domain=0:2, smooth, variable=\x, blue] plot ({\x}, {\x});
            \draw[scale=1, domain=2:4, smooth, variable=\x, blue] plot ({\x}, {4-\x});
            \draw[dashed] (2, 0) -- (2, 2);
            \draw[dashed] (4, 0) -- (4, 0);
            \node at (2, -0.3) {$a$};
            \node at (4, -0.3) {$b$};
            \node at (-0.3, 2) {$A$};
            \end{tikzpicture}
        \end{center}
        \item The particle is most likely to be found at $x = a$.
        \item The probability of finding the particle to the left of $a$ is
        \begin{align*}
            P &= \int_{-\infty}^{a} |\Psi (x, 0)|^2 \dif x\\
            &= \int_{0}^{a} |A (x/a)|^2 \dif x\\
            &= \frac{A^2}{a^2} \int_{0}^{a} x^2 \dif x\\
            &= A^2 \frac{a}{3}\\
            &= \frac{3}{b} \frac{a}{3}\\
            &= \frac{a}{b}.
        \end{align*}
        When $b = a$, we have $P = 1$. When $b = 2a$, we have $P = 1/2$.
        \item The expectation value of $x$ is
        \begin{align*}
            \langle x \rangle &= \int_{-\infty}^{\infty} x |\Psi (x, 0)|^2 \dif x\\
            &= \int_{0}^{a} x |A (x/a)|^2 \dif x + \int_{a}^{b} x |A (b - x)/(b - a)|^2 \dif x\\
            &= \frac{A^2}{a^2} \int_{0}^{a} x^3 \dif x + \frac{A^2}{(b - a)^2} \int_{a}^{b} x (b - x)^2 \dif x\\
            &= A^2 \frac{b(b^3 - 3 a^2 b + 2 a^3)}{12 (b - a)^2}\\
            &= \frac{3}{b} \frac{b(2a + b)}{12}\\
            &= \frac{2a + b}{4}.
        \end{align*}
    \end{enumerate}
\end{sol}

\begin{prob}
    Consider the wave function
    \begin{equation*}
        \Psi (x, t) = A e^{-\lambda |x|} e^{-i \omega t},
    \end{equation*},
    where $A$, $\lambda$, and $\omega$ are positive real constants. (We'll see in Chapter 2 for what potential $(V)$ this wave function satisfies the Schrödinger equation.)
    \begin{enumerate}[label=(\alph*)]
        \item Normalize $\Psi$.
        \item Determine the expectation values of $x$ and $x^2$.
        \item Find the standard deviation of $x$. Sketch the graph of $|\Psi|^2$, as a function of $x$, and mark the points $\langle x \rangle + \sigma$ and $\langle x \rangle - \sigma$, to illustrate the sense in which $\sigma$ represents the "spread" in $x$. What is the probability that the particle would be found outside this range?
    \end{enumerate}
\end{prob}

\begin{sol}
    \begin{enumerate}[label=(\alph*)]
        \item \begin{align*}
            1 &= \int_{-\infty}^{\infty} |\Psi (x, t)|^2 \dif x\\
            &= \int_{-\infty}^{\infty} |A e^{-\lambda |x|} e^{-i \omega t}|^2 \dif x\\
            &= A^2 \int_{-\infty}^{\infty} e^{-2 \lambda |x|} \dif x\\
            &= 2 A^2 \int_{0}^{\infty} e^{-2 \lambda x} \dif x\\
            &= 2 A^2 \frac{1}{2 \lambda}\\
            &= \frac{A^2}{\lambda}.
        \end{align*}
        Thus, we have $A = \sqrt{\lambda}$.
        \item The expectation value of $x$ is
        \begin{align*}
            \langle x \rangle &= \int_{-\infty}^{\infty} x |\Psi (x, t)|^2 \dif x\\
            &= \int_{-\infty}^{\infty} x |A e^{-\lambda |x|} e^{-i \omega t}|^2 \dif x\\
            &= A^2 \int_{-\infty}^{\infty} x e^{-2 \lambda |x|} \dif x\\
            &= 0.
        \end{align*}
        The expectation value of $x^2$ is
        \begin{align*}
            \langle x^2 \rangle &= \int_{-\infty}^{\infty} x^2 |\Psi (x, t)|^2 \dif x\\
            &= \int_{-\infty}^{\infty} x^2 |A e^{-\lambda |x|} e^{-i \omega t}|^2 \dif x\\
            &= A^2 \int_{-\infty}^{\infty} x^2 e^{-2 \lambda |x|} \dif x\\
            &= 2 A^2 \int_{0}^{\infty} x^2 e^{-2 \lambda x} \dif x\\
            &= 2 A^2 \frac{1}{4 \lambda^3}\\
            &= \frac{1}{2 \lambda^2}.
        \end{align*}
        \item The standard deviation of $x$ is
        \begin{align*}
            \sigma &= \sqrt{\langle x^2 \rangle - \langle x \rangle^2}\\
            &= \sqrt{\frac{1}{2 \lambda^2}}\\
            &= \frac{1}{\sqrt{2} \lambda}.
        \end{align*}
        The graph of $|\Psi|^2$ is shown below.
        \begin{center}
            \begin{tikzpicture}
            \draw[->] (-4, 0) -- (4, 0) node[right] {$x$};
            \draw[->] (0, -0.5) -- (0, 2.5) node[above] {$|\Psi(x, t)|^2$};
            \draw[scale=1, domain=-4:4, smooth, variable=\x, blue] plot ({\x}, {exp(-2*abs(\x))});
            \draw[dashed] (0, 1) -- (1, 1);
            \draw[dashed] (0, 1) -- (-1, 1);
            \node at (1, -0.3) {$\sigma$};
            \node at (-1, -0.3) {$-\sigma$};
            \node at (-0.3, 1.3) {$\sqrt{\lambda}$};
            \end{tikzpicture}
        \end{center}
        The probability that the particle would be found outside the range $[-\sigma, \sigma]$ is
        \begin{align*}
            P &= \int_{-\infty}^{-\sigma} |\Psi (x, t)|^2 \dif x + \int_{\sigma}^{\infty} |\Psi (x, t)|^2 \dif x\\
            &= 2 \int_{\sigma}^{\infty} |\Psi (x, t)|^2 \dif x\\
            &= 2 A^2 \int_{\sigma}^{\infty} e^{-2 \lambda x} \dif x\\
            &= 2 A^2 \frac{1}{2 \lambda} e^{-2 \lambda \sigma}\\
            &= e^{-2 \lambda \sigma}\\
            &= e^{-\sqrt{2}}\\
            &\approx 0.2431.
        \end{align*}
    \end{enumerate}
\end{sol}

\begin{prob}
    Why can't you do integration-by-parts directly on the middle expression in Equation 1.29—pull the time derivative over onto $x$, note that $\partial x / \partial t = 0$, and conclude that $d \langle x \rangle / dt = 0$?
\end{prob}

\begin{sol}
    \begin{align}
        \frac{\dif}{\dif t} x \lvert \Psi \rvert^2 &= x \frac{\partial}{\partial t} \lvert \Psi \rvert^2 + \lvert \Psi \rvert^2 \frac{\partial x}{\partial t}\\
        &= x \frac{\partial}{\partial t} \lvert \Psi \rvert^2
    \end{align}
\end{sol}

\begin{prob}
    Calculate $d \langle p \rangle / dt$. Answer:
    \begin{equation*}
        \frac{\dif \langle p \rangle}{\dif t} = \left\langle - \frac{\partial V}{\partial x} \right\rangle.
    \end{equation*}
    This is an instance of \textit{Ehrenfest's theorem}, which asserts that expectation values obey the classical laws.
\end{prob}

\begin{sol}
    \begin{align}
        \frac{\dif \langle p \rangle}{\dif t} &= -i \hbar \frac{\dif}{\dif t} \int_{-\infty}^{\infty} \Psi^* \frac{\partial \Psi}{\partial x} \dif x\\
        &= -i \hbar \int_{-\infty}^{\infty} \left( \frac{\partial \Psi^*}{\partial t} \frac{\partial \Psi}{\partial x} + \Psi^* \frac{\partial}{\partial x} \frac{\partial \Psi}{\partial t}\right) \dif x\\
        &= -i \hbar \int_{-\infty}^{\infty} \left[ \left( -\frac{i \hbar}{2m} \frac{\partial^2 \Psi^*}{\partial x^2} + \frac{i}{\hbar}V \Psi^*\right) \frac{\partial \Psi}{\partial x} +
        \Psi^* \frac{\partial}{\partial x} \left( \frac{i \hbar}{2m} \frac{\partial^2 \Psi}{\partial x^2} - \frac{i}{\hbar}V \Psi \right) \right] \dif x\\
        &= \frac{\hbar^2}{2m} \int_{-\infty}^{\infty}\left( \Psi^* \frac{\partial^3 \Psi}{\partial x^3} - \frac{\partial^2 \Psi^*}{\partial x^2} \frac{\partial \Psi}{\partial x}\right) \dif x +
        \int_{-\infty}^{\infty} \left[V \Psi^* \frac{\partial \Psi}{\partial x} - \Psi^* \frac{\partial}{\partial x} (V \Psi)\right] \dif x\\
        &= \frac{\hbar^2}{2m} \left[ \left(\Psi^* \frac{\partial^2 \Psi}{\partial x^2} \right) \bigg|_{-\infty}^{\infty} - \int_{-\infty}^{\infty} \frac{\partial \Psi^*}{\partial x} \frac{\partial^2 \Psi}{\partial x^2} \dif x -
        \left( \frac{\partial \Psi^*}{\partial x} \frac{\partial \Psi}{\partial x} \right) \bigg|_{-\infty}^{\infty} + \int_{-\infty}^{\infty} \frac{\partial \Psi^*}{\partial x} \frac{\partial^2 \Psi}{\partial x^2} \right]\dif x\\
        &+ \int_{-\infty}^{\infty} \left[V \Psi^* \frac{\partial \Psi}{\partial x} - \Psi^* V \frac{\partial \Psi}{\partial x} - \Psi^* \frac{\partial V}{\partial x} \Psi\right] \dif x\\
        &= \int_{-\infty}^{\infty} - \Psi^* \left[ \frac{\partial V}{\partial x} \right] \Psi \dif x\\
        &= \left\langle - \frac{\partial V}{\partial x} \right\rangle.
    \end{align}
\end{sol}

\begin{prob} \textbf{Problem 1.8}\\
    Suppose you add a constant $V_0$ to the potential energy (by "constant" I mean independent of $x$ as well as $t$). In classical mechanics this doesn't change anything, but what about \textit{quantum mechanics}? Show that the wave function picks up a time-dependent phase factor: $\exp(-i V_0 t / \hbar)$. What effect does this have on the expectation value of a dynamical variable?
\end{prob}

\begin{sol}
    Suppose the wave function $\Psi$ satisfies the Schrödinger equation without the constant $V_0$:
    \begin{equation}
        i \hbar \frac{\partial \Psi}{\partial t} = -\frac{\hbar^2}{2m} \frac{\partial^2 \Psi}{\partial x^2} + V \Psi.
    \end{equation}
    Then, for the wave function $\Psi' = \Psi e^{-i V_0 t / \hbar}$, we have
    \begin{align}
        i \hbar \frac{\partial \Psi'}{\partial t} &= i \hbar \frac{\partial}{\partial t} (\Psi e^{-i V_0 t / \hbar})\\
        &= i \hbar \left( \frac{\partial \Psi}{\partial t} e^{-i V_0 t / \hbar} - \frac{i V_0}{\hbar} \Psi e^{-i V_0 t / \hbar} \right)\\
        &= \left(-\frac{\hbar^2}{2m} \frac{\partial^2 \Psi}{\partial x^2} + V \Psi \right) e^{-i V_0 t / \hbar} + V_0 \Psi e^{-i V_0 t / \hbar}\\
        &= -\frac{\hbar^2}{2m} \frac{\partial^2 \Psi'}{\partial x^2} + (V + V_0) \Psi',
    \end{align}
    which is the Schrödinger equation with the constant $V_0$. 
    Thus, the wave function $\Psi$ picks up a time-dependent phase factor $\exp(-i V_0 t / \hbar)$.
    The expectation value of a dynamical variable is not affected by the phase factor, since the $x$-independent phase factor is canceled out when taking the expectation value.
\end{sol}

\end{document}
